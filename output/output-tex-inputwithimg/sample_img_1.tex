\documentclass{article}
\usepackage{pgfplots}
\pgfplotsset{compat=1.17}

\begin{document}

\begin{tikzpicture}
    \begin{axis}[
        axis lines=left,
        xlabel={Input (capital per worker)},
        ylabel={Output per worker},
        ymin=0, ymax=2.5,
        xmin=0, xmax=3.5,
        xtick=\empty,
        ytick=\empty,
        enlargelimits=false,
        height=8cm,
        width=10cm,
        every axis plot post/.append style={/tikz/domain=0:3.5},
        legend pos=north west
    ]
    
    % Original Production Function
    \addplot [draw=black!50] {sqrt(x)};
    \node at (3.5, 1) {$y = G(N')f(k)$};
    
    % New Production Function Due to Agglomeration
    \addplot [draw=black!20] {1.5*sqrt(x)};
    \node at (3.5, 1.5) {$y = G(N)f(k)$};
    
    \end{axis}
\end{tikzpicture}

External economies of agglomeration as experienced by a firm through a shift in the production function due to an increase in the size of a city ($N$ to $N'$).

\end{document}